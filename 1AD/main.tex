\documentclass[a4paper,11pt]{article}

\usepackage[T1]{fontenc}
\usepackage[utf8]{inputenc}
\usepackage{lmodern}
\usepackage{geometry}
\geometry{margin=1in}
\usepackage{amsmath,amssymb}
\usepackage{graphicx}
\usepackage{booktabs}
\usepackage{hyperref}

% Basic link styling for readability.
\hypersetup{
  colorlinks=true,
  linkcolor=blue,
  citecolor=blue,
  urlcolor=blue
}

\title{DD2434 Advanced Machine Learning\\Assignment 1AD}
\author{Reuben Gezang}
\date{November 2025}

\begin{document}

\maketitle



\section*{D-Level}
\subsection*{Theory 1.D.1}
\subsubsection*{Question 1.1.1}
We start by stating the definition of the Kullback-Leibler divergence:
\begin{equation}
KL(q(Z) \| p(Z|X)) = \mathbb{E}_{q(\boldsymbol{Z})}[\log( \frac{q(Z)}{p(Z|X)})] = \int q(Z) \log( \frac{q(Z)}{p(Z|X)}) dZ
\end{equation}
Now we separete the fraction inside the logarithm:
\begin{equation}
KL(q(Z) \| p(Z|X)) = \int q(Z) \log( \frac{q(Z)p(X)}{p(X,Z)}) dZ = \int q(Z) (\log(q(Z)) - \log(p(Z|X)) dZ
\end{equation}
Note that $p(Z|X) = \frac{p(X,Z)}{p(X)}$ and using this we can rewrite the equation as:
\begin{equation}
KL(q(Z) \| p(Z|X)) = \int q(Z) \log(q(Z)) dZ - \int q(Z) \log(p(X,Z)) dZ + \log(p(X)) \int q(Z) dZ
\end{equation}
Since $q(Z)$ is a probability distribution, we know that $\int q(Z) dZ = 1$. Now we can solve for $\log(p(X))$:
\begin{equation}
\log(p(X)) = KL(q(Z) \| p(Z|X)) - \int q(Z) \log(q(Z)) dZ + \int q(Z) \log(p(X,Z)) dZ
\end{equation}
Combining the logarithm terms, we get:
\begin{equation}
\log(p(X)) = KL(q(Z) \| p(Z|X)) + \mathbb{E}_{q(\boldsymbol{Z})}[\frac{\log(p(X,Z))}{q(Z)}]
\end{equation}
Now we can identify the Evidence lower bound (ELBO)
\begin{equation}
\mathcal{L}(q) = \mathbb{E}_{q(\boldsymbol{Z})}[\frac{\log(p(X,Z))}{q(Z)}]
\end{equation}
and with this we have shown that:
\begin{equation}
\log(p(X)) = \mathcal{L}(q) + KL(q(Z) \| p(Z|X))
\end{equation}
concluding the proof.
\subsubsection*{Question 1.1.2}
KOLLA ÖVER IGEN
In this question we are to describe (in one sentence) how the choice of variational family $q(Z)$ affects
\begin{itemize}
  \item (i) The tightness of the ELBO
  \item (ii) The accuracy of the posterior approximation
\end{itemize}
(i) A more expressive variational family can lead to a tighter ELBO as it can better approximate the true posterior, reducing the KL divergence term.\\
\\
(ii) The choice of variational family directly impacts the accuracy of the posterior approximation, as a limited family may not capture the true posterior's complexity, leading to a less accurate approximation.
\subsection*{Question 1.D.2}
\subsubsection*{1.1.3}
For a mean field assumption and joint distribution
\begin{equation*}
  q(Z_1, Z_2, Z_3) = q_1(Z_1)q_2(Z_2)q_3(Z_3), \quad p(X, Z)
\end{equation*}
Let $q^*_1(Z_1)$ be the $q_1$ that maximizes the ELBO. We want to show that $q_1^*$ satisfies
\begin{equation*}
  \log q_1^*(Z_1) = \mathbb{E}_{-Z_1}[\log p(X, Z)]
\end{equation*}
We can start by inspecting the ELBO:
\begin{equation*}
  \mathcal{L}(q) = \mathcal{L}(q) = \mathbb{E}_{q(\boldsymbol{Z})}[\frac{\log(p(X,Z))}{q(Z)}]=\mathbb{E}_{q}[\log p(X, Z)] - \mathbb{E}_{q}[\log q(Z)]
\end{equation*}
and using the mean field assumption we can rewrite this as:
\begin{equation*}
  \mathcal{L}(q) = \mathbb{E}_{q(Z)}[\log p(X, Z)] - \mathbb{E}_{q(Z)}[\log(q_1(Z_1)q_2(Z_2)q_3(Z_3))] 
\end{equation*}
and by separating the logarithm we get:
\begin{equation*}
  \mathbb{E}_{q(Z)}[\log(q_1(Z_1)q_2(Z_2)q_3(Z_3))] = \mathbb{E}_{q_1(Z_1)}[\log(q_1(Z_1))] + \mathbb{E}_{q_2(Z_2)}[\log(q_2(Z_2))] + \mathbb{E}_{q_3(Z_3)}[\log(q_3(Z_3))]
\end{equation*}
Since we are maximizing w.r.t $q_1(Z_1)$ we can ignore the terms that do not depend on it. Thus we can rewrite the ELBO as:
\begin{equation*}
  \mathcal{L}(q) = \mathbb{E}_{q(Z)}[\log p(X, Z)] - \mathbb{E}_{q_1(Z_1)}[\log(q_1(Z_1))] + C
\end{equation*}
where $C$ is a constant w.r.t $q_1(Z_1)$ (and can thus be ignored). Now we can rewrite the expectation over $q(Z)$ as:
\begin{equation*}
  \mathbb{E}_{q(Z)}[\log p(X, Z)] = \mathbb{E}_{q_1(Z_1)}[\mathbb{E}_{q_2(Z_2)q_3(Z_3)}[\log p(X, Z)]]
\end{equation*}
meaning that we can rewrite the ELBO as:
\begin{equation*}
  \mathcal{L}(q) = \mathbb{E}_{q_1(Z_1)}[\mathbb{E}_{q_2(Z_2)q_3(Z_3)}[\log p(X, Z)]] - \mathbb{E}_{q_1(Z_1)}[\log(q_1(Z_1))] + C
\end{equation*}
Using the fact that $\int_{Z_1} q(Z_1)dZ_1 = 1$ we will now optimize the ELBO w.r.t $q_1(Z_1)$ and with a lagrange multiplier $\lambda$.
\begin{equation}
  \frac{\partial}{\partial q_1(Z_1)} \left( \mathbb{E}_{q_1(Z_1)}[\mathbb{E}_{q_2(Z_2)q_3(Z_3)}[\log p(X, Z)]] - \mathbb{E}_{q_1(Z_1)}[\log(q_1(Z_1))] + \lambda (\int_{Z_1} q(Z_1)dZ_1 - 1) \right) = 0
\end{equation}
giving that
\begin{equation}
  \mathbb{E}_{q_2(Z_2)q_3(Z_3)}[\log p(X, Z)] - \log(q_1(Z_1)) - 1 + \lambda = 0 \rightarrow \log(q^*_1(Z_1)) = \mathbb{E}_{q_2(Z_2)q_3(Z_3)}[\log p(X, Z)] + \lambda - 1
\end{equation}
where $\lambda -1$ is a additive constant that can be ignored when normalizing $q_1^*(Z_1)$. Thus we have shown that:
\begin{equation}
  \log q_1^*(Z_1) = \mathbb{E}_{-Z_1}[\log p(X, Z)]
\end{equation}
as required.
\subsection*{Practice/Implementation - D level (I have chosen 1.D.3)}
\subsubsection*{1.2.4}
The log likelihood of the data is:

\appendix
\section{Additional Details}
Add derivations, extra figures, or ablation studies here.

\end{document}
